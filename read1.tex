%%%%%%%%%%%%%%%%
% switch from arc to beamer
%%%%%%%%%%%%%%%%
\PassOptionsToClass{a4paper,10pt}{article} % for article mode 
\PassOptionsToClass{aspectratio=1610,10pt}{beamer} % for beamer, handout, trans modes
\newcommand*{\BeamerswitchSpawn}[1]{ 
\ShellEscape{... -jobname=\jobname#1 \jobname} %
}
\documentclass{beamerswitch}
\mode<article>{
    \usepackage[top=3cm, bottom=3cm, left=3cm, right=3cm]{geometry}
}
\mode<presentation>{
    \usefonttheme[onlymath]{serif}
}

\usepackage{amsmath, amsthm, amsfonts}
\usepackage{subfiles, bm, hyperref, graphicx, listings}
\usepackage{authblk}

% my cmd
%%%%%%%%%%%%%%%%
% start my commands
%%%%%%%%%%%%%%%%
\newcommand{\lm}{\lambda_\textrm{max}}
\newcommand{\trace}{\mathbf{trace}}
\newcommand{\diag}{\mathbf{diag}}
\newcommand{\model}[1]{(\textbf{#1})}
\newcommand{\mx}{\mathbf{\max}\;}
\newcommand{\mn}{\mathbf{\min}\;}
\newcommand{\st}{\mathrm{s.t.\;}}
\newcommand{\ex}{\mathbb E}
\newcommand{\dx}{\;\bm dx}
\newcommand{\pr}{\mathbb P}
\newcommand{\id}{\mathbb I}
\newcommand{\bp}{\mathbb P}
\newcommand{\be}{\mathbb E}
\newcommand{\bi}{\mathbb I}
\newcommand{\bxi}{\bm \xi}
\newcommand{\bo}{\bm \omega}
\newcommand{\va}{\mathbf{Var}}
\newcommand{\dif}{\mathbf{d}}
\newcommand{\minp}[2]{\min\{#1, #2\}}
\newcommand{\intp}{\mathbf{int}}
\newcommand{\apex}{\mathbf{apex}}
\newcommand{\conv}{\mathbf{conv}}
%%%%%%%%%%%%%%%%
% finish my commands
%%%%%%%%%%%%%%%%

% theorem environments
\theoremstyle{definition}
% my code style
%%%%%%%%%%%%%%%%
% start my commands
%%%%%%%%%%%%%%%%
\lstdefinestyle{codestyle}{
    basicstyle=\ttfamily\footnotesize,
    tabsize=2
    \scriptsize
}
\title{Notes on Stochastic Programming}

\begin{document}

\author[1]{\small Chuwen Zhang}
\author[1]{\small Jingyuan Yang}
\affil[1]{\footnotesize Stochastic Programming Reading Group}

\maketitle
\section{Preface}

This monograph records the reading notes for the subject ``stochastic programming'' starting from Spring 2022. The goal is to increase the authors' familiarity in this fascinating field.
\section{Mathematical Background}
The part is based on \cite{shapiro_lectures_2014}, \cite{birge_introduction_2011} and so forth.
\subsection{Basic Convex Analysis}

Firstly, we introduce some definitions.
\begin{table}[h!]
    \centering
    \begin{tabular}{l|l}
        \(K_C\)          & recession cone  to set \(C\) \\
        \(C^*\)          & polar of cone to set \(C\)   \\
        \(\mathcal N_C\) & normal cone to set \(C\)     \\
        \(\mathcal R_C\) & radial cone to set \(C\)     \\
        \(\mathcal T_C\) & tangent cone to set \(C\)
    \end{tabular}
    \caption{A summary of convex sets}
\end{table}
\begin{definition} The basic convex sets.

    \begin{enumerate}[i]
        \item The recession cone \(K_C\) of the set \(C\) is formed by vectors \(h\) such that for any \(x \in C\) and any \(t > 0\) it follows that \(x + t\cdot h \in C\).
    \end{enumerate}

\end{definition}
\begin{theorem} The theorems for the recession cone.
    \begin{enumerate}[i]
        \item The \(K_C\) is convex if \(C\) is convex,
        \item and is closed if the set C is closed.
        \item The convex set C is bounded iff. \(K_C = \{0\}\)
    \end{enumerate}
\end{theorem}

\begin{theorem} The theorems for the recession cone.
    \begin{enumerate}[i]
        \item The \(K_C\) is convex if \(C\) is convex,
        \item and is closed if the set C is closed.
        \item The convex set C is bounded iff. \(K_C = \{0\}\)
    \end{enumerate}
\end{theorem}


\section{Two-Stage Problems}

\begin{frame}[allowframebreaks]{Introduction}
    To understand the difficulties of multistage (integer) programs, we first look at the structural properties of the value functions.

    Consider the two-stage problem,
    \begin{align*}
        \min_{x \in \mathcal X }~ & c^T x+ \mathcal Q (x)        \\
        \st                       & A x \leq b, x \in \mathcal X
    \end{align*}
    where the second stage recourse, the value function \(\mathcal Q (x) \) is defined as the expectation with a recourse variable \(y\).
    \begin{align*}
        \mathcal Q (x) & = \ex Q(x, \bxi)                                \\
                       & \bxi(\omega)    = (T, W, h )(\omega)            \\
                       & T(\bo) x+W(\bo) y=h(\bo), y \geq 0, \forall \bo \\
        Q(x, \bxi)     & = \min_{y \in \mathcal Y} q^T y                 \\
    \end{align*}

    \paragraph{Remark}
    \begin{itemize}
        \item Assume random variable \(\bo \) living on some probability space, \((\Omega,\mathcal F, P)\)
        \item The first stage variable \(x\) is made before a realization of \(\bxi\), i.e., the ``here-and-now'' ones.
        \item The second stage variable \(y\), to be precise, should be \(\bm y(\omega)\) that actually depends on \(\bo\)
    \end{itemize}
\end{frame}
\begin{frame}[allowframebreaks]{Analysis on the value function}


    We first assume \((x, y)\) are continuous, for example, \(\mathcal X, \mathcal Y\) are convex. Define the function the support function \(s_{q}(\chi)\) of set \(\Pi(q)\), we notice,
    \begin{align*}
        \chi                    & = h - Tx                                                        \\
        \Pi(q)                  & \doteq \left\{\pi: W^{\top} \pi \leq q\right\}                  \\
        s_{q}(\chi)             & \doteq \inf \left\{q^{\top} y: W y=\chi, \quad y \geq 0\right\} \\
        \Rightarrow s_{q}(\chi) & =\sup _{\pi \in \Pi(q)} \pi^{\top} \chi
    \end{align*}

    Obviously,
    \begin{theorem} The value function \(Q(x, \bxi) = s_q(\chi)\), furthermore
        \begin{enumerate}
            \item Q is a homogeneous polyhedron function supporting \(\Pi(q)\)
            \item Also, the subdifferential of \(Q\) could also be defined.
                  \begin{align*}
                      \partial Q\left(x_{0}, \bxi\right)=-T^{\top} \mathcal D \left(x_{0}, \bxi\right)
                  \end{align*}
                  where
                  \begin{align*}
                      \mathcal  D (x, \bxi) = \pi ^* \doteq \arg \max _{\pi \in \Pi(q)} \pi^{\top}(h-T x)
                  \end{align*}
        \end{enumerate}
    \end{theorem}
    \section{Analysis on Value Function}
    Now we inspect the existence of \(Q\). Firstly, we introduce the definition,
    \begin{definition}
        Classification of recourse.
        The recourse problem is,
        \begin{enumerate}[i]
            \item (Fixed) if the matrix \(W(\bo) = W\) is fixed (not random).
            \item (Complete) if the system \(\{y: Wy = \chi, y \ge 0\}\) has a solution for every \(\chi\)
            \item (Simple) if both complete and fixed
            \item (Relatively complete) relatively complete if for every feasible \(x\),
                  the feasible set of the second-stage problem is nonempty for almost everywhere (a.e.) \(\bm \omega \in \Omega\).
        \end{enumerate}
    \end{definition}

    For a recourse to be complete, \(\Pi(q)\) must be bounded, i.e., the recession cone \(\Pi(0) = \{0\}\).

\end{frame}
\begin{frame}[allowframebreaks]{Two-stage integer program}
    For the integer second stage case, first look at the example in \cite{schultz_stochastic_2003}

    \[
        \min _{\left(y, y^{\prime}\right)}\left\{q^{T} y+q^{\prime T} y^{\prime}: T x+W y+W^{\prime} y^{\prime}=h(\omega), y \in Z _{+}^{\bar{n}_{2}}, y^{\prime} \in R _{+}^{n_{2}^{\prime}}\right\}
    \]
\end{frame}
\begin{frame}[allowframebreaks]
    \bibliography{robust}
    \bibliographystyle{apalike}
\end{frame}
\end{document}