\documentclass[a4pper,11pt]{article}
\usepackage{lipsum} %This package just generates Lorem Ipsum filler text. 
\usepackage{graphicx}
\usepackage{natbib}
\usepackage{geometry}
\usepackage[english]{babel}
\usepackage[utf8]{inputenc}
\usepackage[linesnumbered,ruled,vlined]{algorithm2e}
\usepackage[noend]{algpseudocode}
\usepackage[usenames]{color}
\usepackage{amsmath,amsthm,amssymb,amsfonts}
\usepackage{dsfont}
\usepackage{mathrsfs}
\usepackage{tabu}
\usepackage{multirow}
\usepackage{enumerate}
\usepackage{listings}
\usepackage{subfigure}
\usepackage{float}
\usepackage{authblk}
\usepackage{bm}
\usepackage{calligra}
% my cmd
%%%%%%%%%%%%%%%%
% start my commands
%%%%%%%%%%%%%%%%
\newcommand{\lm}{\lambda_\textrm{max}}
\newcommand{\trace}{\mathbf{trace}}
\newcommand{\diag}{\mathbf{diag}}
\newcommand{\model}[1]{(\textbf{#1})}
\newcommand{\mx}{\mathbf{\max}\;}
\newcommand{\mn}{\mathbf{\min}\;}
\newcommand{\st}{\mathrm{s.t.\;}}
\newcommand{\ex}{\mathbb E}
\newcommand{\dx}{\;\bm dx}
\newcommand{\pr}{\mathbb P}
\newcommand{\id}{\mathbb I}
\newcommand{\br}{\mathbb R}
\newcommand{\bp}{\mathbb P}
\newcommand{\be}{\mathbb E}
\newcommand{\bi}{\mathbb I}
\newcommand{\bxi}{\bm \xi}
\newcommand{\bo}{\bm \omega}
\newcommand{\va}{\mathbf{Var}}
\newcommand{\dif}{\mathbf{d}}
\newcommand{\minp}[2]{\min\{#1, #2\}}
\newcommand{\intp}{\mathbf{int}}
\newcommand{\apex}{\mathbf{apex}}
\newcommand{\conv}{\mathbf{conv}}
%%%%%%%%%%%%%%%%
% finish my commands
%%%%%%%%%%%%%%%%

% theorem environments
\newtheorem{thm}{Theorem}[section]
\newtheorem{defn}[thm]{Definition}
\newtheorem{prop}[thm]{Proposition}
\newtheorem{cor}[thm]{Corollary}
\newtheorem{lem}[thm]{Lemma}
\newtheorem{remark}[thm]{Remark}
% my code style
%%%%%%%%%%%%%%%%
% start my commands
%%%%%%%%%%%%%%%%
\title{Notes on Stochastic Programming}

\begin{document}

\author[1]{\small Chuwen Zhang}
\author[1]{\small Jingyuan Yang}
\affil[1]{\footnotesize Stochastic Programming Reading Group}
\maketitle

\section{Intro}
\paragraph{Expected value function}
So far, we usually consider problem with expected value $\be [F(x,\omega)]$.
\begin{enumerate}
	\item Pros: let us focus on long-term performance (irrespective of the fluctuations of specific outcome realization), justified by the Law of Large Numbers
	\item Cons: leave out consideration of risks of loss.
\end{enumerate}
\paragraph{Classical approach.} use scalar transformation $u: \br \to\br$, also called \textit{disutility function}.
\begin{enumerate}
	\item assumed to be nondecreasing and convex
	\item reformulated as the problem
	$$
	\min_{x\in \mathcal X} \be [u(F(x,\omega))]
	$$
	\item difficulties: specifying the disutility function
\end{enumerate}
Modern approach: use risk measures $Z(\omega)=F(x,\omega)$


\paragraph{Mean-risk models.} characterize $Z_x(\omega)$ by mean $\be [Z]$ and risk $\mathbb D[Z]$.
\begin{enumerate}
	\item ``efficient" solutions: for a given value of the mean, they minimize the risk, and for a given value of risk they maximize the mean.
	\item how to find: by techniques of multiobjective optimization
	$$\rho[Z]=\be [Z]+c\mathbb D[Z]$$
	$c$ plays the role of the price of risk.  By varying the value of $c$, we can generate a large ensemble of efficient soluitons. 
	\end{enumerate}
	
\section{Common risk functions $\mathbb D[Z]$}
\begin{enumerate}
	\item $\mathbb{V}ar [Z]$
	\item Central semideviations
		\begin{enumerate}
			\item Upper semideviation of order $p$: 
			$$\sigma^+_p[Z]=(\be[(Z-\be[Z])^p_+])^{1/p}.$$ 
			Appropriate for minimization problems. It is aimed at penalization of an excess of $Z_x$ over its mean. $Z_x$ represents a cost here.
			\item  Lower semideviation of order $p$: $\sigma^-_p[Z]=(\be[(\be[Z]-Z)^p_+])^{1/p}$.
			\item A special case: mean absolute deviation
			$$\sigma_1(Z)=\be [Z-\be [Z]]$$
			We have $\sigma^+_1[Z]=\sigma^-_1[Z]=1/2\sigma_1[Z]$.
		\end{enumerate}
	\item Value-at-risk: 
			$$V@R_\alpha(Z)=\inf\{t:P(Z>t)\leq \alpha\}.$$
	However, chance constraints are often nonconvex. Since $V@R_\alpha(Z)\leq 0\Leftrightarrow \be[\boldsymbol 1_{(0,\infty)}(Z)]\leq \alpha$, a possible approach is to constructing a convex approximation of the expected value.
	\item Average Value-at-Risk (Conditional Value-at-Risk)
	$$AV@R_\alpha(Z)=\inf\{t+\alpha^{-1}\be[Z-t]_+\}$$
	is convex and gives a conservative approximation of the chance constraint.
	\item Weighted Mean Deviations from Quantiles
		$$q_\alpha[Z]=\be[\max [(1-\alpha)(V@R_\alpha(Z)-Z),\alpha (Z-V@R_\alpha(Z))]]$$
		\begin{enumerate}
			\item In the case of minimization, we use $q_{1-\alpha}[Z_x]$ as $\mathbb D[Z]$.
			If $c\in [0,\alpha^{-1}]$ and $Z_x$ is convex in x, then the problem is convex.
			\item Related with $AV@R$: 
			$$AV@R_\alpha(Z)=\be[Z]+\frac{1}{\alpha}q_{1-\alpha}[Z].$$
			\item Related with absolute semideviation:
			$$\sigma^+_1[Z]=\max_{\alpha\in[0,1]}q_\alpha[Z].$$
		\end{enumerate}
\end{enumerate}
\section{Coherent risk measures}


\paragraph{Risk Measure.} $\rho(Z)$ maps $Z$ into the extended real line for $Z\in \mathcal Z=\mathcal L_p(\Omega,\mathcal F,P)$ where $p\in (1,+\infty]$.

A risk measure $\rho(Z)$ is proper if $\rho(Z)>-\infty$ for all $Z\in \mathcal Z$ and the domain 
$$\text{dom}\rho=\{Z\in \mathcal Z: \rho(Z)\leq \infty\}$$
is nonempty.
\paragraph{Axioms associated with a risk measure $\rho$}
\begin{enumerate}
	\item For $Z,Z'\in \mathcal Z$, $Z\succeq Z'\Leftrightarrow Z(\omega)\geq Z'(\omega)$ for a.e. $\omega \in \Omega$.
	\item We assume the smaller the realization of $Z$, the better.
\end{enumerate}

A risk measure $\rho$ is \textit{coherent} if it satisfies the following conditions:
\begin{enumerate}
	\item (Convexity)
	$$\rho(tZ+(1-t)Z')\leq t\rho(Z)+(1-t)\rho(Z')$$
	for all $Z,Z'\in \mathcal Z$ and $t\in[0,1]$.
	\item (Monotonicity) If $Z,Z'\in \mathcal Z$ and $Z\succeq Z'$, then $\rho(Z)\geq \rho(Z')$.
	\item (Translation equivariance) For a scalar $a\in \br$ and $Z\in \mathcal Z$, then $\rho (Z+\alpha)=\rho(Z)+\alpha$.
	\item (Positive homogeneity) If $t>0$ and $Z\in \mathcal Z$, then $\rho(tZ)=t\rho (Z)$.
\end{enumerate}
\textit{Examples.} $AV@R_\alpha(Z)$.

\textit{Remark.} We also can define a coherent risk measure $\mathcal Q(Z)=\rho(-Z)$, if the random outcome represents a reward.

\paragraph{Basic duality results.}
With each space $\mathcal Z= \mathcal L_p(\Omega,\mathcal F,P)$ is associated its dual space $\mathcal Z^*=\mathcal L_q(\Omega,\mathcal F,P)$ where $q\in (1,+\infty]$ is such that $1/p+1/q=1$. For $Z\in \mathcal Z$ and $\zeta\in \mathcal Z^*$, we can define their scalar product
$$<\zeta,Z>=\int_\Omega \zeta(\omega)Z(\omega) dP(\omega).$$
Furthermore, the conjugate function $\rho^*$ of $\rho$ is defined as
$$\rho^*(\zeta)=\sup_{Z\in \mathcal Z}\{<\zeta,Z>-\rho(Z)\}.$$

\begin{thm}
Suppose that $\rho$ is convex, proper, and lower semicontinuous, then $\rho^{**}=\rho$ with $\zeta\in \mathfrak A=dom(\rho^*)$. 
\begin{enumerate}
	\item The monotonicity of $\rho$ holds iff every $\zeta\in \mathfrak A$ is nonnegative.
	\item The translation equivariance of $\rho$ holds iff $\int_\Omega \zeta dP=1$ for every $\zeta\in\mathfrak A$
	\item The positive homogeneity OF $\rho$ holds iff $\rho$ is the support function of $\mathfrak A$, i.e. 
	$$\rho(Z)=\sup_{\zeta\in \mathfrak A}<\zeta,Z> \quad \forall Z\in \mathcal Z.$$
\end{enumerate}
\end{thm}
If $\rho$ is coherent, then $\mathfrak A$ can be viewed as a subset of  the set of a probability density function
$$\mathfrak B =\{\zeta \in \mathcal Z^*:\int_\Omega \zeta(\omega)dP(\omega)=1,\zeta\succeq 0\}.$$ Consequently, for any $\zeta\in\mathfrak A$, $<\zeta,Z>=\be \zeta[Z]$. So $\rho(Z)$ can be written in the form
$$\rho(Z)=\sup_{\zeta\in \mathfrak A}\be \zeta[Z].$$
If $\rho$ is positively homogenous, then $\mathfrak A=\partial \rho(0)$ and $\mathfrak A$ is weakly closed. Furthermore, if $\rho$ is continuous, then $\mathfrak A$ is bounded.

\begin{prop}If $\rho$ is convex and monotone, then $\rho$ is continuous and subdifferentiable on $\mathcal Z$.
\end{prop}
\begin{prop}If $\rho$ is convex, monotone and translation equivariant, and the domain of $\rho$ has a nonempty interior. Then $\rho$ is finite valued and continuous on $\mathcal Z$.
\end{prop}
\begin{thm}
$\rho$ is real valued coherent risk measure iff there exists a convex bounded and weakly closed set $\mathfrak A\subset \mathfrak B$ such that $\rho(Z)=\sup_{\zeta\in \mathfrak A}<\zeta,Z> \quad \forall Z\in \mathcal Z.$
\end{thm}


\end{document}

