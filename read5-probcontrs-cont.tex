\documentclass[a4pper,11pt]{article}
\usepackage{lipsum} %This package just generates Lorem Ipsum filler text. 
\usepackage{graphicx}
\usepackage{natbib}
\usepackage{geometry}
\usepackage[english]{babel}
\usepackage[utf8]{inputenc}
\usepackage[linesnumbered,ruled,vlined]{algorithm2e}
\usepackage[noend]{algpseudocode}
\usepackage[usenames]{color}
\usepackage{amsmath,amsthm,amssymb,amsfonts}
\usepackage{dsfont}
\usepackage{mathrsfs}
\usepackage{tabu}
\usepackage{multirow}
\usepackage{enumerate}
\usepackage{listings}
\usepackage{subfigure}
\usepackage{float}
\usepackage{authblk}

% my cmd
%%%%%%%%%%%%%%%%
% start my commands
%%%%%%%%%%%%%%%%
\newcommand{\lm}{\lambda_\textrm{max}}
\newcommand{\trace}{\mathbf{trace}}
\newcommand{\diag}{\mathbf{diag}}
\newcommand{\model}[1]{(\textbf{#1})}
\newcommand{\mx}{\mathbf{\max}\;}
\newcommand{\mn}{\mathbf{\min}\;}
\newcommand{\st}{\mathrm{s.t.\;}}
\newcommand{\ex}{\mathbb E}
\newcommand{\dx}{\;\bm dx}
\newcommand{\pr}{\mathbb P}
\newcommand{\id}{\mathbb I}
\newcommand{\bp}{\mathbb P}
\newcommand{\be}{\mathbb E}
\newcommand{\bi}{\mathbb I}
\newcommand{\bxi}{\bm \xi}
\newcommand{\bo}{\bm \omega}
\newcommand{\va}{\mathbf{Var}}
\newcommand{\dif}{\mathbf{d}}
\newcommand{\minp}[2]{\min\{#1, #2\}}
\newcommand{\intp}{\mathbf{int}}
\newcommand{\apex}{\mathbf{apex}}
\newcommand{\conv}{\mathbf{conv}}
%%%%%%%%%%%%%%%%
% finish my commands
%%%%%%%%%%%%%%%%

% theorem environments
\newtheorem{thm}{Theorem}[section]
\newtheorem{defn}[thm]{Definition}
\newtheorem{cor}[thm]{Corollary}
\newtheorem{lem}[thm]{Lemma}
\newtheorem{remark}[thm]{Remark}
% my code style
%%%%%%%%%%%%%%%%
% start my commands
%%%%%%%%%%%%%%%%
\title{Notes on Stochastic Programming}

\begin{document}

\author[1]{\small Chuwen Zhang}
\author[1]{\small Jingyuan Yang}
\affil[1]{\footnotesize Stochastic Programming Reading Group}
\maketitle

\section{Overview}

Recall chance constrained SP:
\begin{align}
    \min & ~ \mathbb{E}[f(x, Z)]                                       \\
    \st  & ~ \mathbb{P}\left[g_{j}(x, Z) \leq 0, j \in J\right] \ge p,
\end{align}

We have some different strategies to deal with this. The above case is called ``joint'' cc. Alternatively, one may have,

\[
    \mathbb{P}\left[g_{j}(x, Z) \leq 0 \right] \ge p_j, j \in J,
\]

which is referred to as individuals: when \(|J| = 1\), both reduce to the ``single'' case.

We now introduce some different perspectives on the cc problem.
\section{Uncertainty Set and Robust Optimization}
For simplicity assume \(|J| = 1\), follow the principle of RO, one find measurable subsets \(\mathcal Z\) such that \(\pr(Z\in \mathcal Z) \ge p\), then we have,
\begin{align}
                                       & \mathcal X = \{x: \mathbb{P}\left[g(x, Z) \leq 0\right] \ge p\}       \\
    \label{eq.cc.ro}   \Leftrightarrow & ~\{x: g(x, Z) \leq 0, \forall Z\in\mathcal Z, ~\pr(\mathcal Z)\ge p\}
\end{align}

So in the method of RO, one finds equivalent \(\Leftrightarrow\) or approximation \(\Leftarrow\), often \emph{safe convex approximation}, cf. \eqref{eq.cc.ro} to the original cc.

For uncertainty set \(\mathscr{U}\), we always assume the following approximation
\begin{align}
    \mathscr{U} =\big\{\begin{bmatrix}c^{T}    & d \\ A & b \end{bmatrix}
    =\underbrace{\begin{bmatrix}c_{0}^{T}    & d_{0} \\A_{0} & b_{0}\end{bmatrix}}_{\text { nominal data } D_{0}} %
    +\sum_{l=1}^{L} \zeta_{l} \underbrace{\begin{bmatrix} c_l^T   & d_l \\  A_{l} & b_{l}\end{bmatrix}}_{\textrm {basic }}; %
    \zeta \in Z \subset R ^{L}\big\}
\end{align}
By assume \((c, A, d, b)^l\) is constant and the uncertainty is captured by the \(\textbf{perturbation set}\) \(\mathscr{Z} \subseteq \mathbb R^L\)

Consider a generic CC,
\begin{align*}
    \mx & f(x)                           \\
    \st & \pr(a^Tx \le b) \ge 1-\epsilon
\end{align*}

Given \(f\) convex the problem is generally non-convex. Our goal is to find a \(\textbf{safe approximation}\) to CC, for example, a \textcolor{red}{stronger} yet \textcolor{red}{tractable} convex approximation with function \(g\) such that,
\begin{align}
    g(x,a,b,\epsilon) \le 0 \Rightarrow & \pr(a^Tx \le b) \ge 1-\epsilon
\end{align}

Using the PS notation:
\begin{align}
                                                        & \pr\left[\sum_i \xi_i (a_i^Tx - b_i) \le b_0 - a_0 ^T x \right] \ge 1-\epsilon \\
                                                        & \textrm{Let } z_i =  (a_i^Tx - b_i), z_0 = b_0 - a_0 ^T x                      \\
    \label{eq:ro.cc.basic} \Leftrightarrow \quad        & \pr\left[\sum_i z_i \xi_i \le z_0\right] \ge 1-\epsilon                        \\
    \label{eq:ro.cc.basic_reverse}\Leftrightarrow \quad & \pr\left[\sum_i z_i \xi_i \ge z_0\right] \le \epsilon
\end{align}

We continue our discussion with equation \eqref{eq:ro.cc.basic}, \eqref{eq:ro.cc.basic_reverse} in the subsequent sections.
\subsection{Safe Approximation, Basic Example}

In this section we discuss the case where perturbation is bounded in \([-1, 1]\) with \(0\) mean. We introduce 2 basic theorem to start off.

\begin{thm}\label{tm:ro.cc.1}
    Let \(z_l, l = 1, \dots, L\) be deterministic coefficients and i.i.d. R.V.s \(\{\xi_i\}_{i=1}^L\) such that
    \(\ex \xi_i = 0, \xi_i \in [-1, 1], i = 1, \dots, L\), \(\forall \Omega \ge 0\), we have,
    \begin{equation*}
        \pr(z^T\xi \ge \Omega \|z\|_2) \le \exp\left(-\frac{\Omega^2}{2}\right)
    \end{equation*}
\end{thm}

\begin{thm}\label{tm:ro.cc.hoef}
    \emph{(Hoeffding's inequality)} Let i.i.d. R.V.s \(\{\xi_i\}_{i=1}^L\) , suppose \(\exists a_i, b_i\) such that \(\pr (\xi - \ex\xi \in [a_i, b_i]) = 1\), let \(s = \xi^T 1\)
    \begin{align*}
        \pr(s - \ex s \ge t)   & \le \exp\left(-\frac{2t^2}{\|b-a\|^2}\right)  \\
        \pr(|s - \ex s| \ge t) & \le 2\exp\left(-\frac{2t^2}{\|b-a\|^2}\right)
    \end{align*}
\end{thm}


By the above two theorem, we have following corollaries.

\begin{cor} \label{coro:ro.cc.tm1}
    In the Chance Constraints, if \(\{\xi_i\}_{i=1}^L\) are i.i.d. R.V.s such that
    \(\ex \xi_i = 0, \xi_i \in [-1, 1], i = 1, \dots, L\). Then if \(z_0 \ge \Omega\sqrt{\sum_iz_i^2}\), we have \(\pr(z^T\xi > z_0) \le \exp(-\Omega^2/2)\doteq \epsilon\)
\end{cor}
\begin{proof}
    \(z_0 \ge \Omega\sqrt{\sum_iz_i^2} \Rightarrow \pr(z^T\xi > z_0) \le \pr(z^T\xi > \sqrt{\sum_iz_i^2}) \le \exp(-\Omega^2/2)\), cf. \ref{tm:ro.cc.1},
\end{proof}

From Corollary \ref{coro:ro.cc.tm1}, we see how safe approximation is done with perturbation and its partial prior information. The logic is, we construct a safe approximation in a way that the probability in the CCs are parameterized in the robust counterpart. One can derive similar results for \(\xi\) restricted in \(\mathscr L_1, \mathscr L_2, \mathscr L_\infty\) balls.


\section{The stochastic dominance}

The first-order stochastic dominance (FSD), is the following,
\[\pr_A (X \ge x) \ge \pr_B(X\ge x) \Leftrightarrow F_A(x) \le F_B(x)\]

By second-order stochastic dominance (SSD),
\begin{align}
    F_{k}(X ; \eta)=\int_{-\infty}^{\eta} F_{k-1}(X ; \alpha) d \alpha \\
    F_{2}(X ; \eta)=\mathbb{E}\left[(\eta-X)_{+}\right]
\end{align}

To captalize on FSD or SSD,  we find a \emph{reference variable} Y, for example, in portfolio, set \(Y\) to a benchmark outcome.
\begin{align}
     & A_{(1)}(Y)=\left\{X \in \mathcal L_{1}(\Omega, \mathcal F , \pr ): X \succeq_{(1)} Y\right\} \\
     & A_{(2)}(Y)=\left\{X \in \mathcal L_{1}(\Omega, \mathcal F , \pr ): X \succeq_{(2)} Y\right\}
\end{align}

It is proved that \(A_{(2)}(\cdot)\) closed and convex in \(L _{1}(\Omega, \mathcal F, \pr )\), and \(A_{(2)}\) is the closed convex hull of \(A_{(1)}\). In general \(A_{(1)}\) is not convex.

In some cases \(A_{(1)}\) can be convex.

See, \cite{dentcheva_optimization_2003}, and follow-ons.

\section{P-Efficient Points}

Should refer to Dentcheva's series of papers for this. For example, starts with \cite{dentcheva_concavity_2000}

\section{Generalized Concavity}
Should refer to Borell, RJB Wets. This part is related to functional, convex and real analysis. Some of the materials are pretty old.

\section{Sampling and Data-Driven Method}
Should refer to \cite{nemirovski_convex_2007}

\bibliography{robust}
\bibliographystyle{apalike}
\end{document}

